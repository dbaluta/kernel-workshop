% vim: set tw=78 sts=2 sw=2 ts=8 aw et ai:
\documentclass{workshop}

% Comentează liniile de mai jos în cazul în care nu există cod de inclus.
\usepackage{code/highlight}
\usepackage{color}        % dacă e folosit highlight
\usepackage{alltt}        % dacă e folosit highlight

\title[Sesssion 8]{Session 8}
\subtitle{Kernel Activities}
\author{Daniel Băluță, Irina Preșa}
\date{July 11, 2012}

\begin{document}

% Arătăm numărul frame-ului
\setbeamertemplate{footline}[frame number]

\frame{\titlepage}

% NB: Secțiunile nu sunt marcate vizual, ci doar apar în cuprins
\section{Interrupts}
\begin{frame}{Comunication with hardware}
\begin{itemize}
\item polling
\begin{itemize}
\item periodically check status of hardware 
\item hardware might not be always active or ready
\end{itemize}
\item interrupts
\begin{itemize}
\item hardware signals processor 
\item reported asynchronously 
\end{itemize}
\end{itemize}
\end{frame}

\begin{frame}{Interrupt types}
\begin{itemize}
\item synchronous interrupts / exceptions
\begin{itemize}
\item produced by the CPU itself
\item programming errors (division by zero)
\item abnormal conditions (page faults)
\end{itemize}
\item asynchronous interrupts / hardware interrupts
\begin{itemize}
\item classical interrupts, generated by devices
\item not associated with any particular process
\end{itemize}
\end{itemize}
\end{frame}

\begin{frame}{Hardware interrupts}
\begin{itemize}
\item enable hardware to signal to the processor
\item hardware device - interrupt controller - CPU
\item Interrupt ReQuest line
\item interrupt handler (ISR)
\begin{itemize}
\item runs in interrupt context
\item MUST run quickly
\item ACK the interrupt receipt
\item request_irq(irq, handler, flags, name, dev)
\item free_irq(irq, dev)
\end{itemize}
\end{itemize} 
\end{frame}

\section{Deffering work}
\begin{frame}{Bottom halves}
\begin{itemize}
\item offloads the work of the top half
\item important for system response and performance
\item run with interrupts enabled
\item implemented via multiple mechanisms
\begin{itemize}
\item softirq
\item tasklet
\item work queues
\item kernel timers
\end{itemize}
\end{itemize}
\end{frame}

\section{Softirq}
\begin{frame}{Softirq}
\begin{itemize}
\item statically allocated at compile time
\item runs in interrupt context
\begin{itemize}
\item TIMER_SOFTIRQ
\item NET_TX_SOFTIRQ
\item NET_RX_SOFTIRQ
\item BLOCK_SOFTIRQ
\item TASKLET_SOFTIRQ
\end{itemize}
\item processing softirqs
\begin{itemize}
\item on return from handling an interrupt
\item ksoftirqd
\end{itemize}
\item run with interrupts enabled and cannot sleep
\item while a handler runs, softirqs are disable on current processor
\item softirqs (even same type) can run in parallel on different processors
\end{itemize}
\end{frame}


\section{Tasklets}
\begin{frame}{Tasklets}
\begin{itemize}
\item are serialized form of softirqs
\begin{itemize}
\item HI_SOFTIRQ
\item TASKLET_SOFTIRQ
\end{itemize}
\item you almost always want to use tasklets
\item DECLARE_TASKLET(name, tasklet_handler, data)
\item tasklet_schedule()
\item a tasklet always runs on the processor that scheduled it
\item tasklet_kill()
\end{itemize}
\end{frame}



\section{Work Queues}
\begin{frame}{Work Queues}
\begin{itemize}
\item deffer work into a kernel thread, runs in process context
\item DECLARE_WORK(name, func, data)
\item schedule_work(work)
\item flush_scheduled_work()
\item custom workqueue with associated kernel thread
\begin{itemize}
\item create_workqueue(name)
\item queue_work(wq, work)
\item flush_workqueue(wq)
\end{itemize}
\end{itemize}
\end{frame}


\section{Kernel Timers}
\begin{frame}{Kernel Timers}
\begin{itemize}
\item tick rate has a frequency of HZ hertz
\item x86 defines HZ to be 100
\item jiffies, number of ticks since the system booted
\item runs in interrupt context, on top of a softirq
\item struct timer_list 
\item add_timer(my_timer)
\item mod_timer(my_timer, when)
\item del_timer(my_timer)
\end{itemize}
\end{frame}


\section{Keywords}
\begin{frame}{Keywords}
\begin{itemize}
\item hardware interrupt
\item software interrupt
\item tasklet
\item workqueue
\item timer
\end{itemize}
\end{frame}

\section{Resources}
\begin{frame}{Resources}
\end{frame}

\section{Questions}

\end{document}
