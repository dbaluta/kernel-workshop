% vim: set tw=78 sts=2 sw=2 ts=8 aw et ai:
\documentclass{workshop}

% Comentează liniile de mai jos în cazul în care nu există cod de inclus.
\usepackage{code/highlight}
\usepackage{color}        % dacă e folosit highlight
\usepackage{alltt}        % dacă e folosit highlight

\title[Session 4]{Session 4}
\subtitle{Kernel Debugging}
\author{Daniel Băluţă, Irina Preşa}
\date{July 05, 2012}

\begin{document}

% Arătăm numărul frame-ului
\setbeamertemplate{footline}[frame number]

\frame{\titlepage}

% NB: Secțiunile nu sunt marcate vizual, ci doar apar în cuprins
\section{Kprintf}
\begin{frame}{Kprintf}
\end{frame}

\section{Understand a Kernel Oops/Panic}
\begin{frame}{..}
\end{frame}

\section{Classical Debug Solutions}
\begin{frame}{...}
\end{frame}

\section{Memory Debug}
\begin{frame}{...}
\end{frame}

\section{Tracepoints and Kprobe}
\begin{frame}{...}
\end{frame}

\section{Netconsole and Netcat}
\begin{frame}{...}
\end{frame}

\section{Keywords}
\begin{frame}{Keywords}
      \begin{itemize}
        \item x
      \end{itemize}
\end{frame}

\section{Resources}

\section{Questions}

\end{document}
