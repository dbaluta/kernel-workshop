% vim: set tw=78 sts=2 sw=2 ts=8 aw et ai:
\documentclass{workshop}

% Comentează liniile de mai jos în cazul în care nu există cod de inclus.
\usepackage{code/highlight}
\usepackage{color}        % dacă e folosit highlight
\usepackage{alltt}        % dacă e folosit highlight
\usepackage{amsmath}

\title[Session 3]{Session 3}
\subtitle{Introduction to Kernel Space and API}
\author{Daniel Băluță, Irina Preșa}
\date{04 July 2012}

\begin{document}

% Arătăm numărul frame-ului
\setbeamertemplate{footline}[frame number]

\frame{\titlepage}

% NB: Secțiunile nu sunt marcate vizual, ci doar apar în cuprins
\section{Intro}

\subsection{Execution Contexts}
\begin{frame}{Execution Contexts}
\begin{itemize}
\item \textbf{User context}
\item \textbf{Interrupt context}
\begin{itemize}
\item an interrupt can only be preempted by another interrupt,\\ not by a process (user context).
\item as interrupts are asynchronous, the code executed in a interrupt context isn't backed by a well defined process context.
\item not having an associated process context (that can be saved and restored on reschedule), the interrupt context code can't
execute operations such as sleep, schedule(), or access the user's memory.
\end{itemize}
\end{itemize}
\end{frame}

\begin{frame}{Handling Interrupts}
\begin{itemize}
\item Phase 1: Critical Actions
\begin{itemize}
\item other interrupts are masked.
\item e.g. communication with the PIC.
\end{itemize}
\item Phase 2: Immediate Actions
\begin{itemize}
\item only the current interrupt is masked.
\end{itemize}
\item Phase 3: Deferrable Actions
\begin{itemize}
\item can be executed later.
\end{itemize}
\end{itemize}
\end{frame}


\subsection{Copy to/from User}
\begin{frame}{Copy to/from User}
\begin{itemize}
\item \textbf{in} pointer to kernel space\\
=$>$ access to protected data.\\
=$>$ solution: copy\_to\_user.
\item \textbf{out} pointer to kernel space\\
=$>$ kernel memory corruption.\\
=$>$ solution: copy\_from\_user.
\end{itemize}
\end{frame}

\section{Kernel API}
\begin{frame}{Kernel API}
\begin{itemize}
\item printk, classical string operations
\item kmalloc/kfree
\begin{itemize}
\item GFP\_KERNEL/ATOMIC
\end{itemize}
\item lists: doubly linked list API
\begin{itemize}
\item list\_add, list\_del, list\_for\_each
\end{itemize}
\item locking
\begin{itemize}
\item semaphore, mutex, spinlock
\item atomic variables (read, add, sub, set)
\item atomic operations on bits
\end{itemize}
\end{itemize}
\end{frame}

\begin{frame}{Kernel API - Lists Example}
\input{code/list}
\end{frame}

\section{Kernel - User Space Communication}

\subsection{Syscalls}

\begin{frame}{Kernel Trap Flow}
  \begin{columns}
    \begin{column}[l]{0.55\textwidth}
        \begin{enumerate}
                \item<5-> kernel\_trap [number] (e.g. 0x80)
                \item<5-> ring 3 -$>$ ring 0
                \item<5-> call trap\_entry (e.g. syscall\_entry)\\
                (tvector[0x80]=syscall\_entry).
                \item<5-> push registers
                \item<5-> handle trap (e.g. syscall handler)
                \item<5-> pop registers
                \item<5-> ring 0 -$>$ ring 3
        \end{enumerate}
    \end{column}
    \begin{column}[l]{0.35\textwidth}
        \begin{itemize}
                \item<2-> ESP0, ESP3, TSS
                \item<3-> Traps Vector 
                \item<4-> Registers
        \end{itemize}
    \end{column}
  \end{columns}
\end{frame}

\begin{frame}{Syscalls}
\begin{itemize}
\item generate trap: int \$0x80 or sysenter (Pentium II)
\item syscall number: eax
\item parameters: ebx, ecx, edx, esi, edi, ebp
\item result: eax
\end{itemize}
\end{frame}

\begin{frame}{Syscalls}
\input{code/sysex}
\end{frame}

\begin{frame}{Syscalls Entry}
\input{code/entry}
\end{frame}

\begin{frame}{Syscalls}
\input{code/save}
\end{frame}

\begin{frame}{Syscalls Return}
\input{code/ret}
\end{frame}

\begin{frame}{Syscalls}
\input{code/rest}
\end{frame}

\subsection{Pseudo-Filesystems}
%TODO: procfs, sysfs
% read, write, seq

\subsection{Device Files}

\section{Keywords}
\begin{frame}{Keywords}
  \begin{columns}
    \begin{column}[l]{0.5\textwidth}
      \begin{itemize}
        \item certificări
      \end{itemize}
    \end{column}
    \begin{column}[l]{0.5\textwidth}
      \begin{itemize}
        \item apropos, man, info
      \end{itemize}
    \end{column}
  \end{columns}
\end{frame}

\section{Resources}
\begin{frame}{Resources}
  \begin{itemize}
  \item lala
  \end{itemize}
\end{frame}

\section{Questions}

\end{document}
