% vim: set tw=78 sts=2 sw=2 ts=8 aw et ai:
\documentclass{workshop}

% Comentează liniile de mai jos în cazul în care nu există cod de inclus.
\usepackage{code/highlight}
\usepackage{color}        % dacă e folosit highlight
\usepackage{alltt}        % dacă e folosit highlight

\title[Session 3]{Session 3}
\subtitle{Introduction to Kernel Space and API}
\author{Daniel Băluță, Irina Preșa}
\date{04 July 2012}

\begin{document}

% Arătăm numărul frame-ului
\setbeamertemplate{footline}[frame number]

\frame{\titlepage}

% NB: Secțiunile nu sunt marcate vizual, ci doar apar în cuprins
\section{Intro}

\subsection{Execution Contexts}
\begin{frame}{Execution Contexts}
\end{frame}

\subsection{Copy to/from User}
\begin{frame}{Copy to/from User}
\end{frame}

\section{Kernel API}

\section{Kernel - User Space Communication}

\subsection{Syscalls}
%TODO: si implementare?

\subsection{Pseudo-Filesystems}
%TODO: procfs, sysfs
% read, write, seq

\subsection{Device Files}

\section{Keywords}

\begin{frame}{Keywords}
  \begin{columns}
    \begin{column}[l]{0.5\textwidth}
      \begin{itemize}
        \item certificări
      \end{itemize}
    \end{column}
    \begin{column}[l]{0.5\textwidth}
      \begin{itemize}
        \item apropos, man, info
      \end{itemize}
    \end{column}
  \end{columns}
\end{frame}

\begin{frame}{Resources}
  \begin{itemize}
  \item lala
  \end{itemize}
\end{frame}

\section{Questions}

\end{document}
