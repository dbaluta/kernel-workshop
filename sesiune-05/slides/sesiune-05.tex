% vim: set tw=78 sts=2 sw=2 ts=8 aw et ai:
\documentclass{workshop}

% Comentează liniile de mai jos în cazul în care nu există cod de inclus.
\usepackage{code/highlight}
\usepackage{color}        % dacă e folosit highlight
\usepackage{alltt}        % dacă e folosit highlight
\usepackage{verbatim}
\usepackage{listings}

\title[Session 6]{Session 5}
\subtitle{Character Device Drivers}
\author{Daniel Băluţă, Irina Preşa}
\date{July 06, 2012}

\begin{document}

% Arătăm numărul frame-ului
\setbeamertemplate{footline}[frame number]

\frame{\titlepage}

% NB: Secțiunile nu sunt marcate vizual, ci doar apar în cuprins
\section{Files in Linux}
\begin{frame}{Everything is a file}

\begin{itemize}
\item regular files
\item directories
\item symbolic links
\item sockets
\item pipes
\item char device files
\item block device files
\end{itemize}
\end{frame}

\begin{frame}{Device files}
      \begin{itemize}
        \item represent I/O devices or pseudo-devices
	\item located in /dev
	\item created by mknod or udev
	\item character devices
	\item block devices
      \end{itemize}
\end{frame}

\section{Kernel devices abstraction}
\begin{frame}{What is a block device?}
      \begin{itemize}
        \item random access
	\item frequent seek operation
	\item large data volume and speed
	\item hard disk, ram disk, cd-rom
	\item  ls -l /dev/sda*
	\begin{itemize} 
	\item brw-rw---- 1 root disk 8, 0 2012-07-05 08:03 /dev/sda
	\item brw-rw---- 1 root disk 8, 1 2012-07-05 08:03 /dev/sda1
	\item brw-rw---- 1 root disk 8, 2 2012-07-05 08:03 /dev/sda2
	\end{itemize}
      \end{itemize}
\end{frame}


\begin{frame}{What is a char device?}
      \begin{itemize}
        \item sequential access
	\item seeking doesn't make much sense
	\item small data volume and speed
	\item keyboard, mouse, serial ports
	\item  ls -l /dev/ttyS0
	\begin{itemize} 
	\item crw-rw---- 1 root dialout 4, 64 2012-07-05 08:03 /dev/ttyS0
	\item crw-rw---- 1 root dialout 4, 65 2012-07-05 08:03 /dev/ttyS1
	\item crw-rw---- 1 root dialout 4, 66 2012-07-05 08:03 /dev/ttyS2
	\end{itemize}
      \end{itemize}
\end{frame}

\section{Char devices main structures}
\begin{frame}{struct inode}
      \begin{itemize}
        \item file system view of a file
	\item major, minor
	\begin{itemize}
		\item struct dev_t  
		\item MKDEV(major, minor)
		\item major, identifies the driver
	\end{itemize}
	\item device type
		\begin{itemize}
		\item char device, struct cdev
		\end{itemize}
      \end{itemize}
\end{frame}

\begin{frame}{struct file}
      \begin{itemize}
        \item mode (read, write)
	\item flags (nonblock, truncate)
	\item f_pos
	\item file operations
	\item device private data
      \end{itemize}
\end{frame}

\begin{frame}{struct file operations}
      \begin{itemize}
	\item open
	\item release
	\item read
	\item write
	\item lseek
	\item ioctl
	\item mmap
      \end{itemize}
\end{frame}

\begin{frame}{Steps for creating a char device driver}
      \begin{itemize}
        \item register_chrdev_region
	\item cdev_init
	\item cdev_add
	\item cdev_del
	\item unregister_chrdev_region
      \end{itemize}
\end{frame}

\section{Synchronization mechanisms}
\begin{frame}{Waiting queues}
      \begin{itemize}
        \item struct wait_queue_head
	\item init_waitqueue_head
	\item wait_event_interruptible
	\item wait_event
	\item wake_up_interruptible
	\item wake_iup
      \end{itemize}
\end{frame}


\section{Keywords}
\begin{frame}{Keywords}
      \begin{itemize}
        \item char, block
	\item struct cdev
	\item inode, file
	\item struct file operations
	\item wait queue
      \end{itemize}
\end{frame}

\section{Resources}

\section{Questions}

\end{document}
