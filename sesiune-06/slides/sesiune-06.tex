% vim: set tw=78 sts=2 sw=2 ts=8 aw et ai:
\documentclass{workshop}

% Comentează liniile de mai jos în cazul în care nu există cod de inclus.
\usepackage{code/highlight}
\usepackage{color}        % dacă e folosit highlight
\usepackage{alltt}        % dacă e folosit highlight

\title[Session 6]{Session 6}
\subtitle{Kernel: Some bits of memory}
\author{Daniel Băluţă, Irina Preşa}
\date{July 09, 2012}

\begin{document}

% Arătăm numărul frame-ului
\setbeamertemplate{footline}[frame number]

\frame{\titlepage}

% NB: Secțiunile nu sunt marcate vizual, ci doar apar în cuprins
\section{Overview}

\begin{frame}{Kernel Pages}
  \begin{itemize}
    \item basic unit of memory management
    \item processor can address bytes, MMU can translate pages
    \item page size is architecture dependant (x86 - 4KB, x86_64 - 8KB)
    \item struct page, consumes 20MB for a system with 4GB RAM
      \begin{itemize}
        \item flags (dirty, locked in  memory)
        \item _count, pages usage count
	\item virtual, page's virtual address
      \end{itemize}
  \end{itemize}
\end{frame}


\begin{frame}{Zones}
  \begin{itemize}
    \item raison d'etre: hardware limitations
    \item architecture dependent
    \item zone DMA, DMA-able pages (\textless 16MB)
    \item zone NORMAL, normally addressable pages (16MB-896MB)
    \item zone HIGHMEM, dynamically mapped pages (\textgreater 896MB)
    \item struct zone
    \begin{itemize}
	\item watermark
	\item name
    \end{itemize}
  \end{itemize}
\end{frame}

\begin{frame}{Getting pages}
  \begin{itemize}
    \item alloc_page(s)
    \item alloc_pages
    \item __get__free_page(s)
    \item __free_pages
     \item free_page(s)
  \end{itemize}
\end{frame}

\section{Memory allocation in Linux Kernel}

\begin{frame}{kmalloc}
  \begin{itemize}
    \item kmalloc(size, flags)
    \item allocate memory in bytes sized chunks
     \item flags
    \begin{itemize}
	\item GFP_ATOMIC, high priority and must no sleep.
	\item GFP_KERNEL, normal allocation and might block.
    \end{itemize}
  \item GFP_KERNEL
	\begin{itemize}
	\item process context, can sleep
	\item need DMA-able memory, can sleep
	\end{itemize}
   \item GFP_ATOMIC
	\begin{itemize}
	\item process context, cannot sleep
	\item interrupt handlers, timers
	\end{itemize}
  \end{itemize}
\end{frame}

\begin{frame}{flags}
\begin{itemize}
\item action modifiers
\begin{itemize}
\item how the memory should be allocated
\item GFP_WAIT, allocator can sleep
\item GFP_IO, allocator can start disk IO
\item GFP_HIGH, allocator can access emergency pools
\end{itemize}
\item zone modifiers
\begin{itemize}
\item from where to allocate memory
\item GFP_DMA
\item GFP_HIGHMEM
\end{itemize}
\item types
\begin{itemize}
\item a combination of actions and zone modifiers 
\item GFP_ATOMIC = GFP_HIGH
\item GFP_KERNEL = GFP_WAIT + GFP_IO + etc
\end{itemize}
\end{itemize}
\end{frame}
\begin{frame}{vmalloc}
  \begin{itemize}
  \item memory virtually contiguous
  \item memory not physically contiguous
   \item vmalloc(size)
  \item vfree
  \end{itemize}
\end{frame}

\begin{frame}{SLAB Layer}
  \begin{itemize}
    \item caching
    \item frequently used data is allocated and freed often
    \item frequent allocation and deallocation can result in memory fragmentation
    \item improve performance if we know some patterns of allocation
    \item divide different objects into groups called caches
     \begin{itemize}
	\item task struct
	\item struct inode
     \end{itemize}
  \end{itemize}
\end{frame}

\begin{frame}{SLAB Allocator Interface}
  \begin{itemize}
  \item kmem_cache_create
  \begin{itemize}
  \item create a cache of given size
  \item slabtop or cat /proc/slabinfo
   \end{itemize}
   \item kmem_cache_destroy
   \item kmem_cache_alloc
   \begin{itemize}
    \item allocate an object from a given cache
   \end{itemize}
   \item kmem_cache_free
  \end{itemize}
\end{frame}

\section{Memory mapping}
\begin{frame}{Memory mapping}
  \begin{itemize}
    \item int mmap(struct file *file, struct vm_area_struct *vma)
    \item maps the given file onto the given address space
    \item called by the mmap system call
    \item reduce the overhead of copying data between user and kernel space
    \item cat /proc/fd/maps or pmap
    \item remap_pfn_range
    \begin{itemize}
    \item maps physical contiguous address space onto virtual address space
    \end{itemize}
  \end{itemize}
\end{frame}




\section{Keywords}

\begin{frame}{Keywords}
      \begin{itemize}
        \item kmalloc
        \item vmalloc
        \item struct page
        \item GFP_KERNEL, GFP_ATOMIC
        \item SLAB
        \item caches
        \item mmap
      \end{itemize}
\end{frame}

\begin{frame}{Resources}
  \begin{itemize}
    \item Linux Kernel Development, R. Love
    \item Understanding the Linux Kernel, D. Bovet
  \end{itemize}
\end{frame}

\section{Questions}

\end{document}
